\documentclass[12pt,a4paper]{article}
\usepackage{ctex}
\usepackage{amsmath,amscd,amsbsy,amssymb,latexsym,url,bm,amsthm}
\usepackage{epsfig,graphicx,subfigure}
\usepackage{enumitem,balance}
\usepackage{wrapfig}
\usepackage{mathrsfs,euscript}
\usepackage[usenames]{xcolor}
\usepackage{hyperref}
\usepackage[vlined,ruled,linesnumbered]{algorithm2e}
\usepackage{array}
\hypersetup{colorlinks=true,linkcolor=black}

\newtheorem{theorem}{Theorem}
\newtheorem{lemma}[theorem]{Lemma}
\newtheorem{proposition}[theorem]{Proposition}
\newtheorem{corollary}[theorem]{Corollary}
\newtheorem{exercise}{Exercise}
\newtheorem*{solution}{Solution}
\newtheorem{definition}{Definition}
\theoremstyle{definition}

\renewcommand{\thefootnote}{\fnsymbol{footnote}}

\newcommand{\postscript}[2]
 {\setlength{\epsfxsize}{#2\hsize}
  \centerline{\epsfbox{#1}}}

\renewcommand{\baselinestretch}{1.0}

\setlength{\oddsidemargin}{-0.365in}
\setlength{\evensidemargin}{-0.365in}
\setlength{\topmargin}{-0.3in}
\setlength{\headheight}{0in}
\setlength{\headsep}{0in}
\setlength{\textheight}{10.1in}
\setlength{\textwidth}{7in}
\makeatletter \renewenvironment{proof}[1][Proof] {\par\pushQED{\qed}\normalfont\topsep6\p@\@plus6\p@\relax\trivlist\item[\hskip\labelsep\bfseries#1\@addpunct{.}]\ignorespaces}{\popQED\endtrivlist\@endpefalse} \makeatother
\makeatletter
\renewenvironment{solution}[1][Solution] {\par\pushQED{\qed}\normalfont\topsep6\p@\@plus6\p@\relax\trivlist\item[\hskip\labelsep\bfseries#1\@addpunct{.}]\ignorespaces}{\popQED\endtrivlist\@endpefalse} \makeatother

\begin{document}
\noindent

%========================================================================
\noindent\framebox[\linewidth]{\shortstack[c]{
\Large{\textbf{Lab11-NP Reduction}}\vspace{1mm}\\
CS214-Algorithm and Complexity, Xiaofeng Gao \& Lei Wang, Spring 2021.}}
\begin{center}
\footnotesize{\color{blue}$*$ Name:\underline{\quad   Haoyi You  \quad  }\quad Student ID:\underline{\quad 519030910193 \quad} \quad Email: \underline{\quad yuri-you@sjtu.edu.cn \quad}}
\end{center}

\begin{enumerate}
    \item We are feeling experimental and want to create a new dish. There are various ingredients we can choose from and we'd like to use as many of them as possible, but some ingredients don't go well with others. If there are $n$ possible ingredients (numbered 1 to $n$), we write down an $n\cdot n$ matrix giving the discord between any pair of ingredients. This discord is a real number between 0.0 and 1.0, where 0.0 means "they go together perfectly" and 1.0 means "they really don't go together." Here's an example matrix when there are five possible ingredients.
    \begin{center}
        \begin{tabular}{|c|ccccc|}
        \hline
             & 1  & 2 & 3 & 4 & 5\\
        \hline
            1 & 0.0 & 0.4 & 0.2 & 0.9 & 1.0\\
            2 & 0.4 & 0.0 & 0.1 & 1.0 & 0.2\\
            3 & 0.2 & 0.1 & 0.0 & 0.8 & 0.5\\
            4 & 0.9 & 1.0 & 0.8 & 0.0 & 0.2\\
            5 & 1.0 & 0.2 & 0.5 & 0.2 & 0.0\\
        \hline
        \end{tabular}
    \end{center}
    In this case, ingredients 2 and 3 go together pretty well whereas 1 and 5 clash badly. Notice that this matrix is necessarily symmetric; and that the diagonal entries are always 0.0. Any set of ingredients incurs a penalty which is the sum of all discord values between pairs of ingredients. For instance, the set of ingredients $(1,3,5)$ incurs a penalty of $0.2+1.0+0.5 = 1.7$. We define the \textsc{Experimental Cuisine} as follows:

        Given $n$ ingredients to choose from, the $n\times n$ discord matrix and integer $k$ and a number $p$,  decide whether there exists a collection of at least $k$ ingredients that has a penalty $\leqslant p$

    Prove that $\textsc{3-SAT}\leq_p\textsc{Experimental Cuisine}$
    \begin{solution}
    \begin{enumerate}
        \item Firstly, we know $\textsc{3-SAT}\leq_p\textsc{Independent-Set}$,so we only need to prove $\textsc{Independent-Set}\leq_p\textsc{Experimental Cuisine}$.
        \item For $\forall$ graph $G=(V,E)$,we wonder whether there exists an independent set $S$ with $|S|\ge k$.Let $n=|V|$,then we can mark the vertices to $\{v_1,v_2,\dots v_n\}$. Then we map the vertices to ingredients one to one. And $\forall i,j$,we define the discord between $i,j$ that 
        \begin{equation}
            discord_{i,j}=\left\{
        \begin{aligned}
        2 ~~&~~if~(v_i,v_j)\in E \\
        1~~ &~~if~(v_i,v_j)\notin E \\
        \end{aligned}
        \right.
        \end{equation}
    \end{enumerate}
    Then we ask if there exists a collection of at least k ingredients that has a penalty $\le \frac{k*(k-1)}{2}$
    \item Here we prove this construction is correct.\\
    If there exists an independent set $S$ with $|S|\ge k$, we choose $S'$ as a subset of $S$ that $|S'|=k$. Obviously, $S'$ is also an independent set. So we can assume $S=k$.\\
    We assume $S=\{v_{l_1},v_{l_2},\dots v_{l_k}\}$, then $\forall i,j,(v_{l_i},v_{l_j})\notin E$, so $discord{l_i,l_j}=1$. We choose $\{l_1,l_2,\dots l_k\}$ as ingredients, then the penalty $\le \frac{k(k-1)}{2}$.
    \item If there exist an ingredients set $L=\{l_1,l_2,\dots l_k\}$ with penalty $\le \frac{k(k-1)}{2}$. Since $\forall x,y,discord_{x,y}\ge 1$, so $\forall i,j\in L,discord_{i,j}=1$. So the $S$ is an independent set.
    \end{solution}
    \item An induced subgraph $G'=(V',E')$ of a graph $G=(V,E)$ is a graph that satisfies $V'\subseteq V$ and $E' =\{(u,v)\in E| u,v\in V'\}$. Given two graphs $G_1=(V_1,E_1)$ and $G_2=(V_2,E_2)$ and an integer $b$, we need to decide whether $G_1$ and $G_2$ have a common induced subgraph $G_c$ with at least $b$ nodes. This problem is called \textsc{Maximum Common Subgraph} (MCS). Prove that MCS is NP-complete. (Hint: reduce from \textsc{INDEPENDENT-SET})
	\begin{solution}
	\begin{enumerate}
	~\par
	    \item If  $\textsc{Independent-Set}\leq_p\textsc{Maximum Common Subgraph}$, then \textsc{Maximum Common Subgraph} will be NP-complete because \textsc{INDEPENDENT-SET} is an NP-complete problem.
	    \item For $\forall$ given graph $G=(V,E)$ and an integer $k$, we wonder whether there a subset $S\subseteq V$ with $|S|\ge k$ and no two vertices in $S$ are adjacent(Independent Set).\\
        Then we construct two graph $G_1=(V,E_1),G_2=(V,E_2)$ that $G,G_1,G_2$ have the same vertices set.$G_1$ is a complete graph, and $E_2=E_1-E$, which means $G_2$ is a complement graph of $G$.\\
        Then for $G_1,G_2$ we only need to decide whether they have a common induced subugraph $G_e=(V',E')$ with at least $k$ nodes.(Maximum Common Subgraph).
        \item Here we prove this change is correct.//
        If there exists $S\subseteq V$ satisfies Independent-Set condition. Then $\forall v_1,v_2\in S,(v_1,v_2)\notin E$. As for $G_1,G_2$, obviously $(v_1,v_2)\in G_1$. Also, $(v_1,v_2)\in G_2$ because  $G_2$ is a complement graph of $G$ and $(v_1,v_2)\notin G$. So we get $S=V'$, then from $V'$ and $G$ we can construct $G_e$.\\
        Similarly, if there exists $G_e$ satisfies MCS condition, the $\forall v_1,v_2\in V',(v_1,v_2)\notin E$, so $S=V'$.
        \item As a result, $\forall$ Independent-Set problem, we can change it to an MCS problem, so $\textsc{Independent-Set}\leq_p\textsc{Maximum Common Subgraph}$.
	\end{enumerate}
    
    \end{solution}
    \item Let us define the $k$-spanning tree as a spanning tree in which each node has a degree $\leqslant k$. Given a graph $G= (V,E)$ and a positive integer $k$, we need to decide whether there exists a $k$-spanning tree in $G$. Prove that this problem is NP-complete. (Hint: reduce from \textsc{HAMILTONIAN-CYCLE})
    \begin{solution}
    \begin{enumerate}
    ~\par
        \item There are two steps. Firstly we prove $\textsc{Hamiltonian-Cycle}\leq_q\textsc{Hamiltonian-Road}$. Secondly we prove $\textsc{Hamiltonian-Road}\leq_q\textsc{k-spanning tree}$. Then we get $\textsc{Hamiltonian-Cycle}\leq_q\textsc{k-spanning tree}$. We know \textsc{Hamiltonian-Cycle} is an NPC problem, so the \textsc{k-spanning tree} is also an NPC problem.
        \item $\textsc{Hamiltonian-Cycle}\leq_q\textsc{Hamiltonian-Road}$.\\
        \begin{enumerate}
            \item For $\forall$ given graph $G$,randomly choose an vertex $v\in V$ and splits it into two vertices $v_1,v_2$. Then we add two vertices $v_3,v_4$ into $V$. Then we construct a new graph $G'=(V',E')$ that $V'=(V-\{v\})\cup\{v_1,v_2,v_3,v_4\}$. $E'=E_1\cup E_2\cup \{(v_1,v_3),(v_2,v_4)\}$ with $E_1=\{(v_i,v_j)|v_i,v_j\in V-\{v\} \&  (v_i,v_j)\in E\}$, $E_2=\{(v_i,v_1),(v_i,v_2)|(v_i,v)\in E\}$.\\
            \item Then we can assert there exists a Hamiltonian-Cycle in $G\Longleftrightarrow$  there exists a Hamiltonian-Road in $G'$.
            \item proof:\\
            If there exists a Hamiltonian-Cycle $C$ in $G$, assume $C$ begins and ends at $v$ that $C=(v,u_1\dots,u_{n-1},v)$ . So $R=(v_3,v_1,u_1,\dots,u_{n-1},v_2,v_4)$ is a Hamiltonian-Road in $G'$.\\
            Similarly, if there exists a Hamiltonian-Road $R$ in $G'$. Since $d(v_3)=d(v_4)=1$, $R$ must begin at $v_3/v_4$ and end at $v_4/v_3$. So assume  $R=(v_3,v_1,u_1,\dots,u_{n-1},v_2,v_4)$, so $C=(v,u_1\dots,u_{n-1},v)$ is a Hamiltonian-Cycle in $G$.
        \end{enumerate}
        \item $\textsc{Hamiltonian-Road}\leq_q\textsc{k-spanning tree}$.
        \begin{enumerate}
            \item  For $\forall$ given graph $G$, there exists a Hamiltonian-Road in $G\Longleftrightarrow$ there exists a 2-spanning tree in $G$.
            \item         Actually, a Hamiltonian-Road in $G$ is also a 2-spanning tree. Then we prove a 2-spanning tree is a road,and from the definition of tree we can get it is a Hamiltonian-Road.
            \item  Firstly, we prove only 2 of the node $v_1,v_n$ with $d(v_1)=d(v_n)=1$ and for other node $v_i$ with $d(v_i)=2$. That is because $\sum_{i=1}^{n}\limits d(v_i)=2*(n-1)$ and $\forall v_i,d(v_i)>0$.
            \item Secondly, we start at a leaf $v_1$ of the tree. Then we can find a node $v_2$ as the neighbor of $v_1$. Then as for two neighbors of $v_2$, one has been visited just now, and the other must have not been visited because the two neighbors of one node must not linked if we delete this node,due to the characteristic of tree.
            \item Lastly,we visit the other neighbor.Then we can recursively do this until having visited all nodes. Then we get a Hamiltonian-Road.
        \end{enumerate}
    \end{enumerate}
    \end{solution}
    \item We define the decision problem of \textsc{Knapsack Problem} as follows:
    
        Given $n$ indivisible objects, each with a weight of $w_i>0$ kilograms and a value $v_i>0$, a knapsack with capacity of $W$ kilograms and a number $k$, decide whether there is a collection of objects that can be put into the knapsack with a total value $V\geqslant k$.
        
    Prove that \textsc{Knapsack Problem} is NP-complete.
    \begin{solution}
    We prove it based on another problem called \textsc{Subset Sum Problem}.Since we have known that \textsc{Subset Sum Problem} is an NP-complete problem, we only need to prove $\textsc{Subset Sum Problem}\leq_p\textsc{Knapsack Problem}$.
    \begin{enumerate}
        \item For $\forall$ given set $A$ with $|A|=N$, the value of the items in $A$ is $s_i$,and the target sum is $S$.
        \item Then as for a \textsc{Knapsack Problem}, there are N objects, and the $i^{th}$ object has $\omega_i=s_i,v_i=s_i$, capacity $W=S$, and then we need to decide whether there is a collection of objects that can be put into the knapsack with a total value $V\ge S$.\\
        \item Here we prove the above two problems are equal.
        If there exists a subset satisfies the sum is $S$, and also choose the same objects in \textsc{Knapsack Problem}, obviously the value is $S$ and the weight is also $S$ it satisfies the knapsack property.\\
        If there exists a set of objects that satisfies the knapsack property. Since for every objects, the weight is equal to the value, the sum of weight is equal to the sum of value. However, total value $V\ge S$ and total weight $W\le S$, so $V=W=S$. So the sum of subset is $S$.
    \end{enumerate}
    \end{solution}
\end{enumerate}


\newpage


%========================================================================
\end{document} 