\documentclass[12pt,a4paper]{article}
\usepackage{ctex}
\usepackage{amsmath,amscd,amsbsy,amssymb,latexsym,url,bm,amsthm}
\usepackage{epsfig,graphicx,subfigure}
\usepackage{enumitem,balance}
\usepackage{wrapfig}
\usepackage{mathrsfs,euscript}
\usepackage[usenames]{xcolor}
\usepackage{hyperref}
\usepackage[vlined,ruled,linesnumbered]{algorithm2e}
\hypersetup{colorlinks=true,linkcolor=black}

\newtheorem{theorem}{Theorem}
\newtheorem{lemma}[theorem]{Lemma}
\newtheorem{proposition}[theorem]{Proposition}
\newtheorem{corollary}[theorem]{Corollary}
\newtheorem{exercise}{Exercise}
\newtheorem*{solution}{Solution}
\newtheorem{definition}{Definition}
\theoremstyle{definition}

\renewcommand{\thefootnote}{\fnsymbol{footnote}}

\newcommand{\postscript}[2]
 {\setlength{\epsfxsize}{#2\hsize}
  \centerline{\epsfbox{#1}}}

\renewcommand{\baselinestretch}{1.0}

\setlength{\oddsidemargin}{-0.365in}
\setlength{\evensidemargin}{-0.365in}
\setlength{\topmargin}{-0.3in}
\setlength{\headheight}{0in}
\setlength{\headsep}{0in}
\setlength{\textheight}{10.1in}
\setlength{\textwidth}{7in}
\makeatletter \renewenvironment{proof}[1][Proof] {\par\pushQED{\qed}\normalfont\topsep6\p@\@plus6\p@\relax\trivlist\item[\hskip\labelsep\bfseries#1\@addpunct{.}]\ignorespaces}{\popQED\endtrivlist\@endpefalse} \makeatother
\makeatletter
\renewenvironment{solution}[1][Solution] {\par\pushQED{\qed}\normalfont\topsep6\p@\@plus6\p@\relax\trivlist\item[\hskip\labelsep\bfseries#1\@addpunct{.}]\ignorespaces}{\popQED\endtrivlist\@endpefalse} \makeatother

\begin{document}
\noindent

%========================================================================
\noindent\framebox[\linewidth]{\shortstack[c]{
\Large{\textbf{Lab00-Proof}}\vspace{1mm}\\
CS214-Algorithm and Complexity, Xiaofeng Gao, Spring 2021.}}
\begin{center}
%\footnotesize{\color{red}$*$ If there is any problem, please contact TA Haolin Zhou.}

% Please write down your name, student id and email.
\footnotesize{\color{blue}$*$ Name: \underline{Haoyi You}  \quad\quad \quad Student ID: \underline{519030910193}\quad\quad \quad Email:  \underline{yuri-you@sjtu.edu.cn}}
\end{center}

\begin{enumerate}
    \item
    Prove that for any integer $n>2$, there is a prime $p$ satisfying $n<p<n!$. {\color{blue}(Hint: consider a prime factor $p$ of $n!-1$ and prove by contradiction)}
    \begin{proof}
        Assume there exists n satisfying that there is no integer t with $n<t<n!$. Obviously, integer $k=n!-1$ is not a prime.Let $k=p_1^{{\alpha}_1}p_2^{{\alpha}_2}...p_l^{{\alpha}_l}$,with $p_i(1\leq i\leq l)$ are all prime number. From the assumption we know that for any $i$,$p_i \le n$,but it is certain that $(n!-1,p_i)=(-1,p_i)=1$. So the assume fails. QED.
        
    \end{proof}

    \item
    Use the minimal counterexample principle to prove that for any integer $n\ge 7$, there exists integers $i_n\ge 0$ and $j_n\ge 0$, such that $n = i_n \times 2 + j_n \times 3$.
    \begin{proof}
    First,we know $7=2\times 2+3\times 1$,$8=2\times 4+3\times 0$.
    
    Assumption: There exist integers that do not satisfy the given condition, assume $n_1\geq 9$ be the smallest one of them. 
    
    However, let $n_2=n_1-2\geq 7$.If $n_2$  satisfies the condition and there exists integers $i_{n_2}$ and $j_{n_2}$, such that $n_2=i_{n_2}\times 2+j_{n_2}\times 3$.We can get $n_1=(i_{n_2}+1)\times 2+j_{n_2}\times 3$.
    
    As a result,$n_2$ does not satisfy the condition, either.But we have assumed $n_1$ be the smallest one and we find $n_2\leq n_1$.From the minimal counterexample principle,the assumption fails.QED.
    \end{proof}

    \item
    Suppose the function $f$ be defined on the natural numbers recursively as follows: $f(0)=0$, $f(1)=1$, and $f(n)=5f(n-1)-6f(n-2)$, for $n\geq 2$. Use the strong principle of mathematical induction to prove that for all $n\in N$, $f(n)=3^n-2^n$. 
    \begin{proof}
    	Assumption: $f(n)=3^n-2^n$.
        \par First,when $n=1$ and $n=2$,the assumption is valid.
        \par Then,we suppose when $n\leq k$ the assumption is valid. 
        \par So for  $n\leq k+1$
        \begin{equation}
        \begin{split}
        f(k+1)=5f(k)-6f(k-1)=5*(3^k-2^k)-6*(3^{k-1}-2^{k-1})
        \\=3^{k-1}*(5*3-6)+2^{k-1}*(6-5*2)=9*3^{k-1}-4*2^{k-1}=3^{k+1}-2^{k+1}.
        \end{split}
        \end{equation}
        According to the principle of mathematical induction,the assumption is valid.QED.
    \end{proof}

    \item
    An $n$-team basketball tournament consists of some set of $n\geq2$ teams. Team $p$ beats team $q$ iff $q$
does not beat $p$, for all teams $p\neq q$. A sequence of distinct teams $p_{1}$, $p_{2}$,..., $p_{k}$, such that team $p_{i}$ beats team $p_{i+1}$ for $1\leq i<k$ is called a ranking of these teams. If also team $p_{k}$ beats team $p_{1}$, the ranking is called a \emph{k-cycle}. 

Prove by mathematical induction that in every tournament, either there is a ``champion" team that beats every other team, or there is a 3-cycle. 
    \begin{proof}
        Assumption: for any $n\geq2$ team,either there is a ``champion" team that beats every other team, or there is a 3-cycle. 
        \par First,when n=2,team p beats team q, p is the ``champion" team.
        \par Then, we suppose when $n\leq k$ the assumption is valid.
        \par So for  $n\leq k+1$,there are teams $p_{1}$, $p_{2}$,..., $p_{k+1}$.For first k of these teams,the assumption is valid. 
        \par If there is a 3-cycle,there is also a  3-cycle when we add the team $p_{k+1}$.
        \par If there is a ``champion" team, we let $p_1$ beats the other k-1 teams. If $p_1$ also beats $p_{k+1}$,$p_1$ is also the ``champion" team. 
        \par Otherwise, consider the relations between $p_{k+1}$ and $p_i(2\leq i\leq k)$. If $p_{k+1}$ beats all of them, $p_{k+1}$ is the ``champion" team. Otherwise there exists m that $p_m$ beats $p_{k+1}$,so we know that $p_1$ beats $p_m$,$p_m$ beats $p_{k+1}$,$p_{k+1}$ beats $p_1$,which is a 3-cycle. 
        \par According to the principle of mathematical induction,the assumption is valid.QED.    
    \end{proof}

\end{enumerate}

\vspace{20pt}

\textbf{Remark:} You need to include your .pdf and .tex files in your uploaded .rar or .zip file.

%========================================================================
\end{document}
